% Options for packages loaded elsewhere
\PassOptionsToPackage{unicode}{hyperref}
\PassOptionsToPackage{hyphens}{url}
%
\documentclass[
  11pt,
  oneside]{article}

\usepackage{amsmath,amssymb}
\usepackage{setspace}
\usepackage{iftex}
\ifPDFTeX
  \usepackage[T1]{fontenc}
  \usepackage[utf8]{inputenc}
  \usepackage{textcomp} % provide euro and other symbols
\else % if luatex or xetex
  \usepackage{unicode-math}
  \defaultfontfeatures{Scale=MatchLowercase}
  \defaultfontfeatures[\rmfamily]{Ligatures=TeX,Scale=1}
\fi
\usepackage{lmodern}
\ifPDFTeX\else  
    % xetex/luatex font selection
\fi
% Use upquote if available, for straight quotes in verbatim environments
\IfFileExists{upquote.sty}{\usepackage{upquote}}{}
\IfFileExists{microtype.sty}{% use microtype if available
  \usepackage[]{microtype}
  \UseMicrotypeSet[protrusion]{basicmath} % disable protrusion for tt fonts
}{}
\makeatletter
\@ifundefined{KOMAClassName}{% if non-KOMA class
  \IfFileExists{parskip.sty}{%
    \usepackage{parskip}
  }{% else
    \setlength{\parindent}{0pt}
    \setlength{\parskip}{6pt plus 2pt minus 1pt}}
}{% if KOMA class
  \KOMAoptions{parskip=half}}
\makeatother
\usepackage{xcolor}
\usepackage[top=2.5cm,bottom=2.5cm,left=2.5cm,right=2.5cm,headheight=15pt,footskip=1.25cm]{geometry}
\setlength{\emergencystretch}{3em} % prevent overfull lines
\setcounter{secnumdepth}{-\maxdimen} % remove section numbering
% Make \paragraph and \subparagraph free-standing
\makeatletter
\ifx\paragraph\undefined\else
  \let\oldparagraph\paragraph
  \renewcommand{\paragraph}{
    \@ifstar
      \xxxParagraphStar
      \xxxParagraphNoStar
  }
  \newcommand{\xxxParagraphStar}[1]{\oldparagraph*{#1}\mbox{}}
  \newcommand{\xxxParagraphNoStar}[1]{\oldparagraph{#1}\mbox{}}
\fi
\ifx\subparagraph\undefined\else
  \let\oldsubparagraph\subparagraph
  \renewcommand{\subparagraph}{
    \@ifstar
      \xxxSubParagraphStar
      \xxxSubParagraphNoStar
  }
  \newcommand{\xxxSubParagraphStar}[1]{\oldsubparagraph*{#1}\mbox{}}
  \newcommand{\xxxSubParagraphNoStar}[1]{\oldsubparagraph{#1}\mbox{}}
\fi
\makeatother


\providecommand{\tightlist}{%
  \setlength{\itemsep}{0pt}\setlength{\parskip}{0pt}}\usepackage{longtable,booktabs,array}
\usepackage{calc} % for calculating minipage widths
% Correct order of tables after \paragraph or \subparagraph
\usepackage{etoolbox}
\makeatletter
\patchcmd\longtable{\par}{\if@noskipsec\mbox{}\fi\par}{}{}
\makeatother
% Allow footnotes in longtable head/foot
\IfFileExists{footnotehyper.sty}{\usepackage{footnotehyper}}{\usepackage{footnote}}
\makesavenoteenv{longtable}
\usepackage{graphicx}
\makeatletter
\def\maxwidth{\ifdim\Gin@nat@width>\linewidth\linewidth\else\Gin@nat@width\fi}
\def\maxheight{\ifdim\Gin@nat@height>\textheight\textheight\else\Gin@nat@height\fi}
\makeatother
% Scale images if necessary, so that they will not overflow the page
% margins by default, and it is still possible to overwrite the defaults
% using explicit options in \includegraphics[width, height, ...]{}
\setkeys{Gin}{width=\maxwidth,height=\maxheight,keepaspectratio}
% Set default figure placement to htbp
\makeatletter
\def\fps@figure{htbp}
\makeatother
% definitions for citeproc citations
\NewDocumentCommand\citeproctext{}{}
\NewDocumentCommand\citeproc{mm}{%
  \begingroup\def\citeproctext{#2}\cite{#1}\endgroup}
\makeatletter
 % allow citations to break across lines
 \let\@cite@ofmt\@firstofone
 % avoid brackets around text for \cite:
 \def\@biblabel#1{}
 \def\@cite#1#2{{#1\if@tempswa , #2\fi}}
\makeatother
\newlength{\cslhangindent}
\setlength{\cslhangindent}{1.5em}
\newlength{\csllabelwidth}
\setlength{\csllabelwidth}{3em}
\newenvironment{CSLReferences}[2] % #1 hanging-indent, #2 entry-spacing
 {\begin{list}{}{%
  \setlength{\itemindent}{0pt}
  \setlength{\leftmargin}{0pt}
  \setlength{\parsep}{0pt}
  % turn on hanging indent if param 1 is 1
  \ifodd #1
   \setlength{\leftmargin}{\cslhangindent}
   \setlength{\itemindent}{-1\cslhangindent}
  \fi
  % set entry spacing
  \setlength{\itemsep}{#2\baselineskip}}}
 {\end{list}}
\usepackage{calc}
\newcommand{\CSLBlock}[1]{\hfill\break\parbox[t]{\linewidth}{\strut\ignorespaces#1\strut}}
\newcommand{\CSLLeftMargin}[1]{\parbox[t]{\csllabelwidth}{\strut#1\strut}}
\newcommand{\CSLRightInline}[1]{\parbox[t]{\linewidth - \csllabelwidth}{\strut#1\strut}}
\newcommand{\CSLIndent}[1]{\hspace{\cslhangindent}#1}

\usepackage{booktabs}
\usepackage{longtable}
\usepackage{array}
\usepackage{multirow}
\usepackage{wrapfig}
\usepackage{float}
\usepackage{colortbl}
\usepackage{pdflscape}
\usepackage{tabu}
\usepackage{threeparttable}
\usepackage{threeparttablex}
\usepackage[normalem]{ulem}
\usepackage{makecell}
\usepackage{xcolor}
\usepackage{tcolorbox}
\usepackage[hidelinks]{hyperref}
\makeatletter
\@ifpackageloaded{caption}{}{\usepackage{caption}}
\AtBeginDocument{%
\ifdefined\contentsname
  \renewcommand*\contentsname{Tabla de contenidos}
\else
  \newcommand\contentsname{Tabla de contenidos}
\fi
\ifdefined\listfigurename
  \renewcommand*\listfigurename{Listado de Figuras}
\else
  \newcommand\listfigurename{Listado de Figuras}
\fi
\ifdefined\listtablename
  \renewcommand*\listtablename{Listado de Tablas}
\else
  \newcommand\listtablename{Listado de Tablas}
\fi
\ifdefined\figurename
  \renewcommand*\figurename{Figura}
\else
  \newcommand\figurename{Figura}
\fi
\ifdefined\tablename
  \renewcommand*\tablename{Tabla}
\else
  \newcommand\tablename{Tabla}
\fi
}
\@ifpackageloaded{float}{}{\usepackage{float}}
\floatstyle{ruled}
\@ifundefined{c@chapter}{\newfloat{codelisting}{h}{lop}}{\newfloat{codelisting}{h}{lop}[chapter]}
\floatname{codelisting}{Listado}
\newcommand*\listoflistings{\listof{codelisting}{Listado de Listados}}
\makeatother
\makeatletter
\makeatother
\makeatletter
\@ifpackageloaded{caption}{}{\usepackage{caption}}
\@ifpackageloaded{subcaption}{}{\usepackage{subcaption}}
\makeatother

\ifLuaTeX
\usepackage[bidi=basic]{babel}
\else
\usepackage[bidi=default]{babel}
\fi
\babelprovide[main,import]{spanish}
% get rid of language-specific shorthands (see #6817):
\let\LanguageShortHands\languageshorthands
\def\languageshorthands#1{}
\ifLuaTeX
  \usepackage{selnolig}  % disable illegal ligatures
\fi
\usepackage{bookmark}

\IfFileExists{xurl.sty}{\usepackage{xurl}}{} % add URL line breaks if available
\urlstyle{same} % disable monospaced font for URLs
\hypersetup{
  pdftitle={Análisis estadístico del acceso a oportunidades laborales y nivel económico según el grado de discapacidad en Costa Rica},
  pdfauthor={Jose Andrey Prado Rojas; José Carlos Quintero Cedeño; Diego Alberto Vega Víquez},
  pdflang={es},
  pdfkeywords={Discapacidad, Acceso al empleo, Ingreso económico, Grado
de discapacidad, Inclusión laboral, Desigualdad socioeconómica},
  hidelinks,
  pdfcreator={LaTeX via pandoc}}


\title{Análisis estadístico del acceso a oportunidades laborales y nivel
económico según el grado de discapacidad en Costa Rica}
\author{Jose Andrey Prado Rojas \and José Carlos Quintero
Cedeño \and Diego Alberto Vega Víquez}
\date{2025-07-03}

\begin{document}
\maketitle
\begin{abstract}
Este trabajo explora la relación entre el grado de discapacidad y el
acceso a oportunidades laborales y económicas en Costa Rica, utilizando
datos de la Encuesta Nacional sobre Discapacidad (ENADIS 2023). Se parte
del modelo social de la discapacidad y de marcos teóricos sobre
desigualdad estructural y discriminación para analizar las barreras que
enfrentan las personas con discapacidad en el mercado de trabajo. A
través de pruebas estadísticas no paramétricas como Kruskal-Wallis y
chi-cuadrado de independencia, se identifican asociaciones
significativas entre el grado de discapacidad y variables como ingreso
per cápita, condición de actividad, posición en el empleo y cantidad de
horas trabajadas. Los resultados muestran que las personas con mayor
grado de discapacidad presentan desventajas sistemáticas en términos de
participación laboral, calidad del empleo y nivel de ingresos. Se
concluye que existen patrones de exclusión vinculados al grado de
discapacidad que refuerzan desigualdades económicas preexistentes, lo
que evidencia la necesidad de políticas públicas inclusivas que aborden
no solo el acceso, sino también la equidad en las condiciones de empleo.
\end{abstract}


\setstretch{1}
\textbf{Palabras clave}: Discapacidad, Acceso al empleo, Ingreso
económico, Grado de discapacidad, Inclusión laboral, Desigualdad
socioeconómica

\section{Introducción}\label{introducciuxf3n}

La discapacidad constituye una dimensión estructural de la desigualdad
social que afecta a millones de personas en todo el mundo. En contextos
como el costarricense, donde la inclusión es un principio proclamado
pero no siempre concretado, es fundamental comprender cómo varía el
acceso a oportunidades laborales y económicas según el grado de
discapacidad que enfrenta la población. Diversos estudios han demostrado
que las personas con discapacidad no solo enfrentan barreras físicas o
sensoriales, sino también exclusiones sistémicas que limitan sus
posibilidades de participación económica, acceso a empleo formal y
percepción de ingresos dignos. Estas desigualdades no surgen únicamente
de la condición funcional individual, sino de la interacción entre ésta
y un entorno que continúa siendo poco accesible, discriminatorio o
indiferente.

A partir de esta problemática, la presente investigación se plantea la
siguiente pregunta central: ¿Cuáles son las diferencias en el nivel de
ingresos y las oportunidades laborales entre personas con distintos
grados de discapacidad en Costa Rica?

Este proyecto parte de la premisa de que no todas las personas con
discapacidad viven la misma realidad. El grado de discapacidad
---entendido como la severidad de la limitación funcional--- puede
intensificar los efectos de la exclusión laboral y económica. Así,
mientras algunas personas logran insertarse en el mercado laboral en
condiciones relativamente equitativas, otras enfrentan obstáculos
persistentes, como el subempleo, la informalidad o el desempleo
prolongado. La diversidad de trayectorias laborales en función del grado
de discapacidad es una evidencia de que la inclusión no es un fenómeno
binario, sino un continuo complejo influido por múltiples factores
sociales, económicos y políticos.

A partir de datos de la Encuesta Nacional sobre Discapacidad INEC
(2023), esta investigación adopta un enfoque cuantitativo para
identificar patrones en la participación laboral, la posición
ocupacional, las horas trabajadas y el ingreso per cápita del hogar.
Estas dimensiones permiten construir un diagnóstico riguroso sobre las
desigualdades que experimentan las personas con discapacidad en Costa
Rica. En particular, se aplican pruebas estadísticas no paramétricas
(como Kruskal-Wallis) y categóricas (como Chi-cuadrado y V de Cramer)
para detectar asociaciones significativas entre el grado de discapacidad
y diversas variables socioeconómicas.

Los resultados obtenidos aportan evidencia empírica que refuerza la
necesidad de diseñar políticas públicas que no se limiten a promover el
acceso formal al empleo, sino que también aborden la calidad, la
estabilidad y la equidad en las condiciones laborales. Comprender estas
diferencias es un paso clave para avanzar hacia una verdadera inclusión
laboral. Al visibilizar los efectos concretos que tiene el grado de
discapacidad en la vida económica de las personas, este estudio
contribuye a la formulación de estrategias que respondan a las
necesidades específicas de una población históricamente excluida del
desarrollo social.

\section{Metodología}\label{metodologuxeda}

\textbf{Descripción de los datos}

\textbf{Fuente de la información}

La fuente de datos corresponde a la Encuesta Nacional sobre Discapacidad
(ENADIS), aplicada en Costa Rica por el Instituto Nacional de
Estadística y Censos INEC (2023). Esta encuesta proporciona información
representativa a nivel nacional sobre características sociodemográficas,
estado de salud, educación, empleo e ingresos de la población con y sin
discapacidad.

\textbf{Población de estudio}

La población objeto de estudio se limita a las personas con edades entre
18 y 65 años, correspondiente a la población en edad de trabajar. Dentro
de esta población, se analiza a las personas con discapacidad leve,
moderada o severa, comparándolas con aquellas que no reportan
discapacidad.

\textbf{Contexto temporal y espacial de los datos}

Los datos corresponden al año 2023 y tienen cobertura nacional,
incluyendo tanto zonas urbanas como rurales de todas las regiones
planificadas del país. Las estimaciones y análisis se basan en una
muestra compleja, con ponderadores aplicados para representar
correctamente a la población costarricense.

\textbf{Unidad estadística o individuos}

Cada unidad de análisis corresponde a una persona residente en Costa
Rica entre 18 y 65 años, clasificada según su grado de discapacidad,
nivel educativo, condición de actividad, tipo de empleo, horas laboradas
e ingreso del hogar.

\textbf{Variables de interés}

Las variables utilizadas para el análisis son:

\emph{Grado de discapacidad}: sin discapacidad, leve, moderada y severa.

\emph{Condición de actividad}: ocupado, desocupado, fuera de la fuerza
de trabajo.

\emph{Tipo de empleo principal}: empleado, trabajador por cuenta propia,
informal, entre otros.

\emph{Cantidad de horas trabajadas}: categorizadas como menos de 15
horas, de 15 a menos de 40, de 40 a 48, y más de 48 horas semanales.

\emph{Ingreso per cápita del hogar}: calculado dividiendo el ingreso
total del hogar entre la cantidad de personas que lo conforman.

Estas variables se analizan para identificar las diferencias económicas
y laborales asociadas al grado de discapacidad. Para cada una, se
aplicaron pruebas estadísticas no paramétricas y categóricas según
corresponda, considerando los supuestos metodológicos requeridos (como
falta de normalidad en los ingresos y naturaleza categórica de la
mayoría de las variables explicativas).

\textbf{Métodos}

\textbf{Prueba de Kruskal-Wallis}

Esta prueba fue utilizada para determinar si existen diferencias en la
distribución del ingreso per cápita del hogar entre personas
clasificadas según su grado de discapacidad (sin, leve, moderada,
severa). La prueba de Kruskal-Wallis permite comparar tres o más grupos
independientes utilizando rangos en lugar de valores originales, lo cual
es ideal cuando los datos no son normales o tienen valores atípicos
importantes Ostertagová et~al. (2014).

\begin{tcolorbox}[title=Prueba de Kruskal-Wallis]
La prueba de Kruskal-Wallis sirve para ver si las distribuciones de una variable son iguales entre varios grupos. Se basa en rangos y no en los valores originales. El estadístico se calcula con la fórmula:
\begin{center}
$H = \frac{12}{N(N+1)} \sum\limits_{i=1}^{k} \frac{R_i^2}{n_i} - 3(N+1)$
\end{center}
donde \( N \) es el total de observaciones, \( k \) el número de grupos, \( n_i \) el tamaño del grupo \( i \), y \( R_i \) la suma de los rangos del grupo \( i \).
\end{tcolorbox}

La hipótesis nula establece que todas las distribuciones son iguales
entre los grupos, y se rechaza si el valor p es menor al nivel de
significancia (por ejemplo, 0.05), lo que indicaría que al menos un
grupo tiene una distribución distinta de los demás.

\textbf{Prueba Chi-cuadrado de independencia}

Para analizar si existe una relación entre el grado de discapacidad y
otras variables categóricas, como la condición de actividad o el tipo de
jornada laboral, se aplicó la prueba Chi-cuadrado de independencia. Esta
técnica permite evaluar si dos variables categóricas están asociadas o
si son independientes McHugh (2013).

\begin{tcolorbox}[title=Prueba Chi-cuadrado de independencia]
El estadístico de la prueba se calcula con la fórmula:
\begin{center}
$\chi^2 = \sum\limits_{i=1}^r \sum\limits_{j=1}^c \frac{(O_{ij} - E_{ij})^2}{E_{ij}}$
\end{center}
donde \( O_{ij} \) es la frecuencia observada en la celda \( i, j \), y \( E_{ij} \) es la frecuencia esperada, calculada como:

\begin{center}
E_{ij} = \frac{(F_i)(C_j)}{N}
\end{center}

siendo \( F_i \) el total de la fila \( i \), \( C_j \) el total de la columna \( j \), y \( N \) el número total de casos.
\end{tcolorbox}

Cabe destacar que este tipo de análisis ha sido empleado en estudios
como el de Lay-Raby et~al. (2021), donde se utilizaron datos
representativos de Chile para modelar las probabilidades de acceso al
empleo de personas con discapacidad. En ese trabajo se identificaron
variables relevantes como el nivel educativo, el ingreso y los
subsidios, que influyen de forma significativa en las oportunidades
laborales. Aunque en este proyecto no se aplicó un modelo de regresión
logística multinomial, los hallazgos de dicha investigación sirvieron de
apoyo para la interpretación de los patrones encontrados con los métodos
descriptivos y las pruebas de hipótesis seleccionadas.

Cuando la prueba chi-cuadrado detecta una asociación significativa entre
variables categóricas, se hace necesario cuantificar la fuerza de esa
relación. Para este propósito se utilizó la \textbf{V de Cramer}, una
medida que permite interpretar la magnitud del vínculo entre las
variables analizadas.

La V de Cramer tiene la ventaja de ser fácil de interpretar: sus valores
van de 0 a 1, donde 0 indica ausencia de asociación y 1 representa una
relación perfecta. Esta medida es particularmente útil en tablas de
contingencia de cualquier dimensión, y permite comparar el grado de
asociación entre diferentes pares de variables. Aunque no existe una
escala rígida, se suele considerar que valores menores a 0.10 indican
una relación débil, entre 0.10 y 0.29 una relación moderada, y mayores a
0.30 una relación fuerte.

\begin{tcolorbox}[title=V de Cramer]
La V de Cramer es una medida de asociación entre dos variables categóricas. Se utiliza comúnmente después de realizar una prueba chi-cuadrado para cuantificar la intensidad de la relación. Su fórmula es:

\begin{center}
$V = \sqrt{ \frac{\chi^2}{N \cdot (k - 1)} }$
\end{center}

donde:

\begin{itemize}
  \item \( \chi^2 \) es el estadístico de la prueba chi-cuadrado de independencia,  
  \item \( N \) es el número total de observaciones,  
  \item \( k \) es el menor entre el número de filas \( r \) o columnas \( c \): \( k = \min(r, c) \).
\end{itemize}

El valor de \( V \) está acotado entre 0 y 1. Un valor cercano a 0 indica poca o ninguna asociación, mientras que un valor cercano a 1 indica una asociación fuerte.
\end{tcolorbox}

\section{Resultados}\label{resultados}

El análisis de los microdatos de la ENADIS 2023 revela una relación
significativa entre el grado de discapacidad y diversos indicadores de
acceso al empleo, ingreso y calidad laboral. Estos hallazgos empíricos
coinciden con la literatura académica y técnica revisada, tanto en el
contexto costarricense como en estudios realizados en el resto del
mundo. A continuación, se presentan los principales resultados agrupados
en cuatro dimensiones clave: ingreso, condición de actividad y horas
trabajadas, posición ocupacional, y calidad del empleo.

Los ingresos per cápita del hogar varían significativamente según el
grado de discapacidad. La prueba de Kruskal-Wallis aplicada al ingreso
per cápita arrojó un estadístico ( \chi\^{}2 ) = 192.41, con 3 grados de
libertad y un valor p \textless{} 2.2e-16, lo que confirma que las
distribuciones de ingreso no son iguales entre los grupos. Las personas
sin discapacidad tienen las medianas de ingreso más altas, mientras que
estas disminuyen conforme aumenta la severidad de la discapacidad.

Esta diferencia puede observarse gráficamente en la
Figura~\ref{fig-ingreso}, donde se presenta un diagrama de cajas del
ingreso per cápita del hogar (en escala logarítmica) según el grado de
discapacidad.

\begin{figure}

\centering{

\includegraphics[width=0.8\textwidth,height=\textheight]{../res/graficos/ingreso_hogar_vs_grado_discapacidad.png}

}

\caption{\label{fig-ingreso}Distribución del ingreso per cápita del
hogar por grado de discapacidad}

\end{figure}%

Se aprecia que las personas sin discapacidad concentran mayores niveles
de ingreso y menor dispersión relativa. A medida que aumenta el grado de
discapacidad (de leve a severo), la mediana de ingreso disminuye
visiblemente, y la distribución presenta mayor asimetría y presencia de
valores extremos. Además, la media (representada con una ``x'') es
sistemáticamente superior a la mediana, lo que sugiere que existen
algunos ingresos altos que elevan el promedio, especialmente en los
grupos sin discapacidad o con discapacidad leve.

Este resultado es coherente con los hallazgos de Jiménez Lara \& Huete
García (2010) en España, donde se mostró que las personas con
discapacidad no solo tienen menores ingresos, sino que deben asumir
mayores gastos en su vida cotidiana, generando un agravio económico
comparativo frente a la población general. A diferencia de este patrón
de reducción absoluta del ingreso, el estudio de Pu \& Syu (2023) en
Taiwán evidenció un fenómeno diferente: aunque el ingreso total no se
reduce drásticamente tras adquirir una discapacidad, sí cambia su
composición. Las personas con discapacidad pasan a depender menos del
ingreso salarial y más de fuentes pasivas como intereses, renta de
propiedades o transferencias. Este cambio en la composición puede
indicar una mayor exposición a ingresos inestables o no recurrentes,
situación que también podría darse en Costa Rica, aunque se requieren
más estudios para confirmarlo.

El análisis de la condición de actividad mediante la prueba de
Chi-cuadrado reveló diferencias estadísticamente significativas ((
\chi\^{}2 ) = 219.33, gl = 6, p \textless{} 2.2e-16). Las personas con
discapacidad severa presentan una proporción mucho mayor de inactividad
económica, mientras que la tasa de ocupación es más alta en personas sin
discapacidad.

Como se observa en la Figura~\ref{fig-condic}, estas diferencias son
consistentes: la proporción de personas fuera de la fuerza de trabajo
aumenta conforme se agrava el grado de discapacidad, pasando de un 25\%
en personas sin discapacidad a un 45\% en aquellas con discapacidad
severa. Por el contrario, la proporción de personas ocupadas disminuye
notablemente en ese mismo trayecto, bajando de un 71\% a solo un 51\%.
Este patrón es coherente con lo reportado por Ananian \& Dellaferrera
(2024), quienes concluyen que las tasas de participación laboral de
personas con discapacidad son más bajas en la mayoría de países, incluso
cuando estas personas tienen niveles de educación y experiencia
similares a la población sin discapacidad.

\begin{figure}

\centering{

\includegraphics[width=0.8\textwidth,height=\textheight]{../res/graficos/cond_act_seg_grad_disc.png}

}

\caption{\label{fig-condic}Distribución porcentual de la condición de
actividad por grado de discapacidad}

\end{figure}%

Además, la distribución de las horas trabajadas semanalmente presenta
diferencias estadísticamente significativas entre los grupos definidos
por grado de discapacidad (( \chi\^{}2 ) = 173.32, gl = 9, p \textless{}
2.2e-16). Tal como se muestra en la Figura~\ref{fig-horas}, en los
grupos con discapacidad moderada y severa es más común observar jornadas
parciales, especialmente aquellas inferiores a las 40 horas semanales.
Por ejemplo, un 39\% de las personas con discapacidad severa trabaja
menos de 40 horas a la semana, en contraste con un 24\% entre quienes no
reportan discapacidad.

Este patrón indica una reducción en la jornada laboral proporcional a la
severidad de la discapacidad, y sugiere una mayor prevalencia de empleos
de medio tiempo o con condiciones menos estables. Esta tendencia ha sido
documentada por Donelly et~al. (2020), quienes demostraron que los
graduados universitarios con discapacidad, además de recibir menores
ingresos, tienden a trabajar menos horas y a enfrentar trayectorias
laborales más fragmentadas. Asimismo, Henly \& Brucker (2020) señalan
que la reducción de horas suele ir acompañada de menor autonomía y
menores posibilidades de desarrollo profesional, contribuyendo a
perpetuar un ciclo de precariedad.

En el contexto costarricense, Oca \& Andrés (2020) destacan que una
proporción considerable de personas con discapacidad no trabaja debido a
barreras estructurales como la falta de accesibilidad, prejuicios por
parte de empleadores y desconocimiento sobre incentivos legales. Por
tanto, las menores jornadas observadas no necesariamente reflejan una
preferencia voluntaria por parte de las personas con discapacidad, sino
limitaciones impuestas por un entorno laboral poco inclusivo.

\begin{figure}

\centering{

\includegraphics[width=0.8\textwidth,height=\textheight]{../res/graficos/grado_disc_vs_horas_lab.png}

}

\caption{\label{fig-horas}Distribución de las horas laboradas semanales
por grado de discapacidad}

\end{figure}%

El análisis de la variable posición en el trabajo mostró diferencias
claras entre los grupos según el grado de discapacidad (( \chi\^{}2 ) =
114.06, gl = 15, p \textless{} 2.2e-16). Como se observa en la
Figura~\ref{fig-posicion}, las personas con discapacidad tienden a
desempeñarse con mayor frecuencia como trabajadoras por cuenta propia o
como ayudantes sin remuneración, mientras que la categoría de empleados
asalariados es más común entre las personas sin discapacidad.

\begin{figure}

\centering{

\includegraphics[width=0.8\textwidth,height=\textheight]{../res/graficos/pos_trabajo_vs_grad_disc.png}

}

\caption{\label{fig-posicion}Distribución del grado de discapacidad por
posición en el empleo principal}

\end{figure}%

Esta tendencia refleja un mayor grado de informalidad y vulnerabilidad
laboral entre quienes presentan alguna discapacidad, lo cual coincide
con lo planteado por L. A. Schur (2002), quienes señalan que, en
contextos donde el mercado laboral no es inclusivo, las personas con
discapacidad terminan accediendo mayormente a empleos informales,
temporales o de medio tiempo, que suelen estar peor remunerados y
carecer de protección social.

De forma complementaria, Pinilla-Roncancio \& Gallardo (2023) documentan
que en seis países latinoamericanos se observa un patrón similar: las
personas con discapacidad, y especialmente las mujeres y quienes viven
en zonas rurales, enfrentan mayores barreras para acceder al empleo
formal. Estas barreras incluyen prejuicios en los procesos de
contratación, escasa adaptación en los lugares de trabajo y ausencia de
políticas efectivas que incentiven la inclusión laboral. En este
sentido, los resultados costarricenses obtenidos a partir de la ENADIS
2023 refuerzan la evidencia sobre esta exclusión estructural en la
región.

Los resultados anteriores se complementan con el cálculo de la V de
Cramer, una medida estadística utilizada para evaluar la fuerza de la
asociación entre dos variables categóricas. A diferencia de la prueba
Chi-cuadrado, que solo indica si existe una relación significativa, la V
de Cramer permite cuantificar cuán fuerte o débil es esa relación. Sus
valores van de 0 a 1, donde 0 significa ausencia total de asociación y 1
indica una relación perfecta.

En este caso, el grado de discapacidad se contrastó con diferentes
variables categóricas del ámbito laboral. En primer lugar, para la
condición de actividad (ocupado/a, desocupado/a, fuera de la fuerza
laboral), se obtuvo un valor de \(V = 0.1058\), lo cual representa una
asociación débil pero no trivial. Este hallazgo es coherente con lo
observado en la Figura~\ref{fig-condic}, donde se evidencia que la
inactividad económica aumenta conforme se agrava el grado de
discapacidad. Aunque la diferencia es visible, la magnitud moderada del
valor indica que otros factores también contribuyen a la exclusión del
empleo, como la falta de accesibilidad, prejuicios estructurales y
políticas públicas insuficientes ---tal como sugieren Ananian \&
Dellaferrera (2024) y Oca \& Andrés (2020).

Para la variable horas trabajadas semanalmente, el valor de la V de
Cramer fue incluso menor \(V = 0.0939\), indicando una asociación más
débil. En la Figura~\ref{fig-horas}, se aprecia que las personas con
discapacidad severa o moderada tienden a concentrarse en jornadas
parciales (menos de 40 horas), en contraste con quienes no presentan
discapacidad, que predominan en jornadas completas. Este patrón sugiere
que, aunque hay diferencias significativas, la discapacidad por sí sola
no explica completamente la distribución del tiempo laboral, lo cual
coincide con lo documentado por Donelly et~al. (2020) y Henly \& Brucker
(2020) sobre los obstáculos estructurales que afectan la estabilidad y
calidad del empleo.

Finalmente, en el análisis de la posición ocupacional (empleado/a,
patrono/a, cuenta propia, ayudante sin remuneración), el valor de
\(V = 0.0729\) fue el más bajo de los tres casos. Esto indica que, entre
las personas que sí están ocupadas, el tipo de puesto de trabajo no
varía de forma marcada según el grado de discapacidad. Aun así, la
Figura~\ref{fig-posicion} muestra que las personas con discapacidad
tienen mayor presencia en trabajos por cuenta propia y no remunerados,
lo cual puede estar vinculado a procesos de autoexclusión forzada o
empleo informal, como también señalan L. A. Schur (2002) y
Pinilla-Roncancio \& Gallardo (2023) en sus análisis sobre barreras de
acceso al empleo formal.

En conjunto, estos valores de la V de Cramer permiten afirmar que el
grado de discapacidad sí está relacionado con peores condiciones
laborales, pero que la fuerza de dicha relación es limitada, lo cual
refuerza la necesidad de entender la exclusión laboral como un fenómeno
multicausal. La discapacidad no actúa de manera aislada, sino en
interacción con el entorno económico, institucional, educativo y
sociocultural del país.

\section{Conclusiones}\label{conclusiones}

Los hallazgos de esta investigación revelan con claridad que el grado de
discapacidad constituye un factor diferenciador en el acceso a
oportunidades laborales y económicas en Costa Rica. A partir del
análisis de los microdatos de la ENADIS 2023, se evidenció que a mayor
severidad en la discapacidad, menores son los ingresos per cápita, la
participación en el empleo formal y la cantidad de horas trabajadas.
Estos resultados son estadísticamente significativos y respaldados por
pruebas robustas como Kruskal-Wallis y chi-cuadrado de independencia,
que permiten afirmar con fundamento que la desigualdad estructural se
expresa con fuerza entre los distintos grupos según su grado de
discapacidad.

Aunque las medidas de asociación como la V de Cramer muestran que la
fuerza estadística de estas relaciones es débil o moderada, su
relevancia práctica no puede ser ignorada. Las personas con
discapacidad, especialmente aquellas con limitaciones severas, enfrentan
condiciones laborales más precarias, menor acceso a empleos estables y
mayor dependencia de ocupaciones por cuenta propia o informales. Esto se
enmarca en un entorno que no ha logrado adecuarse de manera efectiva a
las necesidades de accesibilidad, adaptación y equidad para esta
población.

En términos de política pública, los resultados respaldan la necesidad
de replantear las estrategias de inclusión laboral desde un enfoque
interseccional y estructural. No basta con promover el empleo entre
personas con discapacidad; es indispensable garantizar que dicho empleo
sea digno, bien remunerado y libre de barreras discriminatorias.
Asimismo, deben fortalecerse los mecanismos de fiscalización, los
incentivos para empleadores inclusivos, y las políticas de accesibilidad
universal tanto en el sector público como en el privado.

Este estudio demuestra que el grado de discapacidad no solo tiene
implicaciones sobre la funcionalidad individual, sino que es un
determinante clave en los resultados socioeconómicos de las personas.
Comprenderlo como un eje de desigualdad estructural permite avanzar
hacia políticas más justas y eficaces. De cara al futuro, se sugiere
profundizar en enfoques multivariados que incluyan variables como sexo,
región y nivel educativo, así como considerar estudios longitudinales
que permitan observar trayectorias laborales en el tiempo. Sólo con
evidencia sólida y políticas basadas en datos será posible construir una
sociedad verdaderamente inclusiva y equitativa.

\section{Agradecimientos}\label{agradecimientos}

Los autores desean expresar su más profundo agradecimiento al Profesor
Ph.D.~Maikol Solís por su acompañamiento durante todo el proceso de
desarrollo del presente trabajo. Su disposición constante para brindar
asistencia en los aspectos técnicos y metodológicos relacionados con el
programa utilizado fue fundamental para el avance y culminación exitosa
de este proyecto.

De igual manera, se extiende un sincero agradecimiento a la Dra. Milena
Castro, quien durante la segunda parte del semestre brindó un valioso
acompañamiento académico. Su orientación en la definición del tema de
investigación, así como sus recomendaciones metodológicas, fueron clave
para enriquecer el enfoque y la solidez del trabajo realizado.

Asimismo, se agradece a los compañeros del curso CA0303: Estadística
Actuarial I, cuyo apoyo y colaboración fueron de gran utilidad a lo
largo de todo el semestre en el que se desarrolló este proyecto.

⸻

\newpage{}

\section{Anexos}\label{anexos}

\subsection{Repositorio en GitHub}\label{repositorio-en-github}

\url{https://github.com/Diego-Vega-Viquez/Bitacoras-CA0303}

\section*{Bibliografía}\label{bibliografuxeda}
\addcontentsline{toc}{section}{Bibliografía}

\phantomsection\label{refs}
\begin{CSLReferences}{1}{0}
\bibitem[\citeproctext]{ref-ameri_disability_2015}
Ameri, M., Schur, L., Adya, M., Bentley, S., McKay, P., \& Kruse, D.
(2015). \emph{The disability employment puzzle: A field experiment on
employer hiring behavior}. National Bureau of Economic Research.
\url{http://www.nber.org/papers/w21560}

\bibitem[\citeproctext]{ref-ananian_study_2024}
Ananian, S., \& Dellaferrera, G. (2024). \emph{A study on the employment
and wage outcomes of people with disabilities} {[}Working Paper 124{]}.
International Labour Organization.
\url{https://doi.org/10.54394/YRCN8597}

\bibitem[\citeproctext]{ref-donelly_income_2020}
Donelly, M., Gordon, S., \& Bowling, A. (2020). Income and employment
equity of graduates with and without disabilities. \emph{Work},
\emph{65}(4), 547-561. \url{https://doi.org/10.3233/WOR-203109}

\bibitem[\citeproctext]{ref-henly_intrinsic_2020}
Henly, M., \& Brucker, D. L. (2020). More than just lower wages:
Intrinsic job quality for college graduates with disabilities.
\emph{Journal of Education and Work}, \emph{33}(5), 474-493.
\url{https://doi.org/10.1080/13639080.2020.1842865}

\bibitem[\citeproctext]{ref-inec_enadis_2023_pdf}
INEC. (2023). \emph{Encuesta {Nacional} sobre {Discapacidad} ({ENADIS}):
{Metodología} y {Resultados} {[}recurso electrónico{]}}. Instituto
Nacional de Estadística y Censos ({INEC}).
\url{https://inec.cr/sites/default/files/documentos/enadis2023-metodologia.pdf}

\bibitem[\citeproctext]{ref-jimenez_agravio_2010}
Jiménez Lara, A., \& Huete García, A. (2010). \emph{Estudio sobre el
agravio comparativo económico que origina la discapacidad}. Ministerio
de Sanidad y Política Social, Universidad Carlos III.
\url{http://www.mscbs.gob.es/}

\bibitem[\citeproctext]{ref-lay_raby_multinomial_2021}
Lay-Raby, N., Fuente-Mella, H. de la, \& Lameles-Corvalán, O. (2021).
Multinomial Logistic Regression to Estimate and Predict the Job
Opportunities for People with Disabilities in Chile. \emph{Information},
\emph{12}(9), 356. \url{https://doi.org/10.3390/info12090356}

\bibitem[\citeproctext]{ref-malo_ocana_genero_2006}
Malo Ocaña, M. Á., \& Dávila Quintana, C. D. (2006). Género,
discapacidad y posición familiar: la participación laboral de las
mujeres con discapacidad. \emph{Cuadernos aragoneses de economía},
\emph{16}(1), 61-82.
\url{https://dialnet.unirioja.es/servlet/articulo?codigo=2005269}

\bibitem[\citeproctext]{ref-mchugh2013chi}
McHugh, M. L. (2013). The Chi-square test of independence.
\emph{Biochemia Medica}, \emph{23}(2), 143-149.
\url{https://doi.org/10.11613/BM.2013.018}

\bibitem[\citeproctext]{ref-oca_retos_2020}
Oca, C. M. de, \& Andrés, H. (2020). Retos para la inclusión laboral de
personas con discapacidad en {Costa} {Rica}. \emph{Economía y Sociedad},
\emph{25}(58), 50-68. \url{https://doi.org/10.15359/eys.25/58.4}

\bibitem[\citeproctext]{ref-ostertagova2014kruskal}
Ostertagová, E., Ostertag, O., \& Kováč, J. (2014). Methodology and
Application of the Kruskal-Wallis Test. \emph{Applied Mechanics and
Materials}, \emph{611}, 115-120.
\url{https://doi.org/10.4028/www.scientific.net/AMM.611.115}

\bibitem[\citeproctext]{ref-pinilla_inequality_2023}
Pinilla-Roncancio, M., \& Gallardo, M. (2023). Inequality in labour
market opportunities for people with disabilities: Evidence for six
Latin American countries. \emph{Global Social Policy}, \emph{23}(1),
67-91. \url{https://doi.org/10.1177/14680181211070201}

\bibitem[\citeproctext]{ref-pu_effects_2023}
Pu, C., \& Syu, H.-F. (2023). Effects of disability on income and income
composition. \emph{PLOS ONE}, \emph{18}(5), e0286462.
\url{https://doi.org/10.1371/journal.pone.0286462}

\bibitem[\citeproctext]{ref-rae_diccionario_2023}
Real Academia Española. (2023). \emph{Diccionario de la lengua española}
(24.ª ed.). Asociación de Academias de la Lengua Española.
\url{https://dle.rae.es}

\bibitem[\citeproctext]{ref-oms_informe_2011}
Salud, O. M. de la, \& Mundial, B. (2011). \emph{Informe mundial sobre
la discapacidad}. Organización Mundial de la Salud.
\url{https://www.who.int/disabilities/world_report/2011/en/}

\bibitem[\citeproctext]{ref-schur_dead_2002}
Schur, L. A. (2002). Dead End Jobs or a Path to Economic Well Being? The
Consequences of Non-Standard Work among People with Disabilities.
\emph{Behavioral Sciences and the Law}, \emph{20}(5), 601-620.
\url{https://doi.org/10.1002/bsl.512}

\bibitem[\citeproctext]{ref-schur_disability_2017}
Schur, L., Han, K., Kim, A., Ameri, M., Blanck, P., \& Kruse, D. (2017).
Disability at {Work}: {A} {Look} {Back} and {Forward}. \emph{Journal of
Occupational Rehabilitation}, \emph{27}(4), 482-497.
\url{https://doi.org/10.1007/s10926-017-9739-5}

\bibitem[\citeproctext]{ref-vargas2013}
Vargas, M. C. (2013). La población con discapacidad en los censos del
siglo {XX} en {Costa} {Rica}. \emph{Población y Salud en Mesoamérica}.
\url{https://doi.org/10.15517/psm.v11i1.10525}

\end{CSLReferences}




\end{document}
